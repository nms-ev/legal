\documentclass[fontsize=12pt,parskip=half] {scrartcl}
\usepackage{lmodern}
\usepackage{textcomp}
\usepackage[shortlabels]{enumitem}
\usepackage[clausemark=forceboth,juratotoc, juratocnumberwidth=2.5em]{scrjura}
\usepackage{eurosym}
\usepackage{csquotes}
\usepackage{pgffor}

\MakeOuterQuote{"}

\newcommand\name{nms e.V.}

\begin{document}

\subject{Vereinssatzung}
\title{\name}
\subtitle{Vereinssatzung für den \name}
\date{\today \\ \version}
\maketitle

\tableofcontents

% Set the head rule for everything but the title and the toc
\pagestyle{body}

\appendix

\section{Allgemeines}
\begin{contract}

    \Clause{title={Name, Sitz, Geschäftsjahr}}

    Der Verein trägt den Namen \name

    Der Verein hat seinen Sitz in Dresden.

    Geschäftsjahr ist das Kalenderjahr.

    Der \name soll in das Vereinsregister Sachsen eingetragen werden.


    \Clause{title={Vereinszweck}}
    \label{clause:vereinszweck}

    Zweck des \name ist die Förderung, Erhaltung und Weiterentwicklung
    des Kulturraumes Dresden, sowie die Förderung der Subkultur der elektronischen Tanzmusik.

    Hierbei wird der Zweck durch die Umsetzung von Veranstaltungen, welche die verschiedenen Kunstformen, hierunter auch Musik, darstellende Künste wie die Schauspielerei, Tanz, Performance, Theater, Kunsthandwerk, Bühnenbildnerei, in sich vereinen, gefördert.

    Der Satzungszweck wird insbesondere verwirklicht durch die eigenständige Planung, Organisation und Durchführung von kulturellen Veranstaltungen, Seminaren und Workshops. Der \name will einen Beitrag leisten zur Erhaltung und Pflege von Kulturwerten sowie zur Erforschung der Kulturlandschaft.

    Der Verein fördert die kulturelle und künstlerische Bildung und Entwicklung kreativer Potentiale im Sinne des lebenslangen Lernens.

    Des Weiteren verfolgt der \name die Förderung des Naturschutzes und der Landschaftspflege im Sinne des Bundesnaturschutzgesetzes und der Naturschutzgesetze der Länder, des Umweltschutzes, einschließlich des Klimaschutzes, des Küstenschutzes und des Hochwasserschutzes.
    Der Satzungszweck wird verwirklicht insbesondere durch den Einsatz überschüssiger Mittel, die als Erlöse aus den in (2) genannten Veranstaltungen anfallen.

    \Clause{title={Mittelverwendung}}
    Mittel der Körperschaft dürfen nur für die satzungsmäßigen Zwecke verwendet werden. Die Mitglieder erhalten keine Zuwendungen aus Mitteln der Körperschaft.

    \Clause{title={Vergütung}}
    Es darf keine Person durch Ausgaben, die dem Zweck der Körperschaft fremd sind, oder durch unverhältnismäßig hohe Vergütungen begünstigt werden.

\end{contract}

\section{Mitgliedschaft}
\begin{contract}
    \Clause{title={Mitgliedsarten}}

    Mitglieder des Vereins können natürliche und juristische Personen werden, die sich den Zielen des Vereins verbunden fühlen.

    Der Verein kann folgende Mitglieder haben:
    \begin{enumerate}[(a)]
        \item Fördermitglieder
        \item Stimmberechtigte Mitglieder
    \end{enumerate}

    \Clause{title={Erwerb der Mitgliedschaft}}

    Fördermitglieder sind Mitglieder, die die Arbeit, Ziele und den Zweck des Vereins in geeigneter Weise, insbesondere durch regelmäßige finanzielle Zuwendungen fördern und unterstützen. Fördermitglied können sowohl natürliche Personen und Personengesellschaften als auch juristische Personen werden. Die Fördermitgliedschaft beginnt durch schriftliche Erklärung einschließlich Einzugsermächtigung der natürlichen Person oder des Vertreters der juristischen Person oder der Personengesellschaft gegenüber dem Verein. Ein Muster der schriftlichen Erklärung wird auf der Homepage des Vereins bereitgestellt. Bei  Minderjährigen ist die Erklärung auch von deren gesetzlichen Vertreter:innen zu unterschreiben. Diese müssen sich durch gesonderte schriftliche Erklärung zur Zahlung der Mitgliedsbeiträge für den Minderjährigen verpflichten. Kann die Erklärung nicht online abgegeben werden, erfolgt sie durch Abgabe des ausgefüllten, unterschriebenen Formulars beim Verein. Fördermitglieder haben kein Stimmrecht; sie werden zu den Mitgliederversammlungen des Vereins geladen, können diesen beiwohnen und sich zu Wort melden.

    Stimmberechtigte Mitglieder fördern und unterstützen die Ideale, Zwecke und Ziele des Vereins durch ihre aktives Mitwirken an Projekten des Vereins. Stimmberechtigtes Mitglied können natürliche und juristische Personen werden. Die Stimmberechtigte Mitgliedschaft beginnt durch schriftliche Erklärung einschließlich Einzugsermächtigung der natürlichen oder juristischen Person gegenüber dem Vorstands des Vereins. Ein Muster der schriftlichen Erklärung wird auf der Homepage des Vereins bereitgestellt. Bei Minderjährigen ist die Erklärung auch von deren gesetzlichen Vertreter:innen zu unterschreiben. Diese müssen sich durch gesonderte schriftliche Erklärung zur Zahlung der Mitgliedsbeiträge für den Minderjährigen verpflichten. Kann die Erklärung nicht online abgegeben werden, erfolgt sie durch Abgabe des ausgefüllten, unterschriebenen Formulars beim Verein. Über eine Aufnahme eines stimmberechtigten Mitgliedes in den Verein entscheidet der Vorstand mit einfacher Mehrheit. Die stimmberechtigte Mitgliedschaft endet automatisch nach 2 Jahren. Ein Antrag zur Erneuerung der stimmberechtigten Mitgliedschaft kann nach Ablauf dieser Zeit jederzeit gestellt werden. Dieser wird wie ein Erst-Mitgliedschafts-Antrag gehandhabt.

    \Clause{title={Rechte und Pflichten der Mitglieder}}

    Die Mitglieder sind verpflichtet
    \begin{enumerate}[(a)]
        \item die Ziele und Interessen des Vereins zu unterstützen,
        \item die Satzung und Vereinsordnungen zu beachten sowie Beschlüsse und Anordnungen der Vereinsorgane zu befolgen,
        \item alle für eine ordnungsgemäße Vereinsverwaltung erforderlichen Daten dem Vorstand oder einer sonst hierzu bevollmächtigten Person zu melden,
        \item den Mitgliedsbeitrag zu entrichten, wenn eine Beitragspflicht besteht.
    \end{enumerate}

    Fördermitglieder haben das Recht, Vorschläge zu Projekten und Aktivitäten des Vereins zu unterbreiten und regelmäßig Informationen zu erhalten. Dies betrifft insbesondere Informationen über die Verwendung der Förderbeiträge.

    Stimmberechtigte Mitglieder fördern die Zwecke und Ziele des Vereins durch ihre aktive und fachliche Mitarbeit an Projekten, Kampagnen und Öffentlichkeitsarbeit. Zudem besteht für sie das Recht, im Rahmen des satzungsmäßigen Zwecks der Mitgliederversammlung Vorschläge zu den Inhalten und der Arbeit des Vereins zu unterbreiten. Auf der Mitgliederversammlung hat jedes stimmberechtigte Mitglied Rede-, Antrags- und Stimmrecht, soweit der Mitgliedsbeitrag entrichtet wurde, es sei denn eine Befreiung des Mitgliedsbeitrages wird beschlossen. Näheres regelt die Vereinsordnung.

    \Clause{title={Beendigung der Mitgliedschaft}}

    Die Mitgliedschaft als Fördermitglied endet
    mit dem Tode, bei juristischen Personen mit Auflösung oder bei Personengesellschaften mit deren Beendigung,
    durch Kündigung der Fördermitgliedschaft, die jederzeit schriftlich gegenüber dem Verein erklärt werden kann,
    durch Einstellung der regelmäßigen Beitragszahlung,
    durch Ausschluss (\ref{ausschluss}).

    Die Mitgliedschaft als stimmberechtigtes Mitglied endet
    \begin{enumerate}[(a)]
        \item mit dem Tode, bei juristischen Personen mit Auflösung oder bei Personengesellschaften mit deren Beendigung,
        \item durch Austrittserklärung, der schriftlich gegenüber dem Verein, der durch den Vorstand vertreten wird, erklärt werden kann,
        \item durch Einstellung der regelmäßigen Beitragszahlung,
        \item durch Ausschluss (\ref{ausschluss}).
    \end{enumerate}

    Ein Mitglied kann wegen eines Verhaltens, das die Belange oder das Ansehen des Vereins schädigt, z.B. wenn es sich gesetzeswidrig oder vereinsschädigend verhält oder in grober Weise gegen die Interessen des Vereins verstößt, oder wegen eines anderen wichtigen Grundes ausgeschlossen werden. Über den Ausschluss stimmberechtigter Mitglieder entscheidet die Versammlung stimmberechtigter Mitglieder mit einer einfachen Mehrheit der anwesenden Stimmen. Ein Ausschluss von stimmberechtigten Mitgliedern ist insbesondere möglich, wenn die Beiträge trotz zweimaliger Mahnung nicht gezahlt werden oder das Mitglied verzogen und seine Anschrift nicht ermittelbar ist. Der Ausschluss von Fördermitgliedern erfolgt in schriftlicher Form durch den Vorstand. Dem betroffenen Mitglied muss vor der Beschlussfassung über den Ausschließungsantrag Gelegenheit zur Stellungnahme gegeben werden.
    \label{ausschluss}

    \Clause{title={Mitgliedsbeiträge}}
    Es wird ein Mitgliedsbeitrag erhoben. Alles Weitere regelt die Beitragsordnung.

    Höhe und Fälligkeit von Beiträgen für stimmberechtigte Mitglieder werden von der Mitgliederversammlung im Rahmen einer Beitragsordnung festgelegt.

    Der Vorstand kann in besonderen Fällen Gebühren und Beiträge ganz oder teilweise erlassen oder stunden.

\end{contract}

\section{Organisation}
\begin{contract}

    \Clause{title={Organe des Vereins}}
    Organe des Vereins sind
    \begin{enumerate}[(a)]
        \item der Vorstand
        \item die Mitgliederversammlung
    \end{enumerate}

    \Clause{title={Der Vorstand}}
    Der Vorstand des Vereins besteht aus dem geschäftsführenden Vorstand i.S.v. \S 26 BGB. Dem Vorstand gehören an: mindestens drei natürliche Personen, die das 18. Lebensjahr vollendet haben und stimmberechtigte Mitglieder des Vereins sind. Die Anzahl der Vorstandsmitglieder muss ungerade sein. Der Vorstand besteht aus der/dem Vorsitzenden und ihren/seinen Stellvertreter:innen.

    Es ist darauf zu achten, dass eine Geschlechtergleichheit innerhalb des Vorstands angestrebt und gefördert wird.

    Gerichtlich und außergerichtlich wird der Verein jeweils durch zwei Mitglieder des Vorstands vertreten. Von den Beschränkungen des \S 181 BGB sind die Mitglieder des Vorstands befreit.

    Der Vorstand ist berechtigt, für bestimmte Aufgabengebiete oder bestimmte Einzelfälle Vollmachten – auch mit Einzelvertretungsmacht – zu erteilen. Näheres regelt die Geschäftsordnung des Vorstands.

    Die Vorstandsmitglieder können hauptamtlich, nebenamtlich und freiberuflich tätig sein und für ihre Tätigkeit eine angemessene Vergütung erhalten.

    \Clause{title={Zuständigkeit des Vorstands; Haftung}}
    Der Vorstand führt die Geschäfte des Vereins. Er ist für alle Angelegenheiten des Vereins zuständig, soweit sie nicht durch die Satzung der Mitgliederversammlung übertragen sind.

    Der Vorstand leitet verantwortlich die Vereinsarbeit. Er kann sich eine Geschäftsordnung geben. Der Vorstand kann für die Geschäfte der laufenden Verwaltung eine:n Geschäftsführer:in bestellen.

    Ist eine Willenserklärung gegenüber dem Verein abzugeben, so genügt die Abgabe gegenüber eines Mitglieds des Vorstands. Dieses ist zur ordnungsgemäßen Weiterleitung verpflichtet.

    Der Vorstand kann Richtlinien zur Förderung oder Durchführung von Projekten durch den Verein festlegen; darüber hinaus zählt zu seinen Aufgaben insbesondere
    \begin{enumerate}[(a)]
        \item Leitung, Tagesgeschäft, Verein und operative Führung des Betriebes, inkl. Personalführung und strategische Weiterentwicklung
        \item Beschlussfassung über die Förderung oder Durchführung von Projekten,
        \item Vorbereitung und Einberufung der Mitgliederversammlung, Aufstellung der Tagesordnung,
        \item Ausführung der Beschlüsse der Mitgliederversammlung,
        \item Vorbereitung des Haushaltsplans, Buchführung und Erstellung des Jahresberichts,
        \item Entscheidung über die Aufnahme von Mitgliedern.
    \end{enumerate}
    Näheres zum buchführerischen Tagesgeschäft des Vereins regelt die Geschäftsordnung des Vorstands.

    Die Haftung des Vorstands ist im Innenverhältnis auf Vorsatz und grobe Fahrlässigkeit begrenzt.

    Der Vorstand ist gegenüber der Mitgliederversammlung zur Rechenschaft verpflichtet und erteilt Auskunft durch Rechnungslegung und Tätigkeitsbericht.

    Der Vorstand erstellt und aktualisiert die Vereinsordnung des Vereins.


    \Clause{title={Wahl und Amtsdauer des Vorstands}}
    Die Mitglieder des Vorstands werden jedes Jahr aus der Mitgliederversammlung heraus
    gewählt. Jedes stimmberechtigte Mitglied, das natürliche Person ist und das 18. Lebensjahr
    vollendet hat, kann sich zur Wahl aufstellen lassen. Eine Wiederwahl ist möglich. Näheres zum Aufstellungs- und Wahlprozess regelt die Vereinsordnung.

    \Clause{title={Sitzungen und Beschlüsse des Vorstands}}
    Vorstandssitzungen finden jährlich mindestens einmal statt. Die Einladung zu Vorstandssitzungen erfolgt durch den Vorstandsvorsitzenden per E-Mail unter Einhaltung einer Einladungsfrist von mindestens 14 Tagen. Näheres regelt die Geschäftsordnung.

    Der Vorstand fasst seine Beschlüsse mit einfacher Mehrheit, insofern \ref{abstimmungsart} nicht zur Geltung kommt.

    Beschlüsse des Vorstands können bei Eilbedürftigkeit auch elektronisch oder fernmündlich gefasst werden, wenn alle Vorstandsmitglieder und die Geschäftsführung ihre Zustimmung zu diesem Verfahren elektronisch oder fernmündlich erklären. Elektronisch oder fernmündlich gefasste Vorstandsbeschlüsse sind schriftlich niederzulegen, von allen Vorstandsmitgliedern zu unterzeichnen und innerhalb von sieben Tagen an die Geschäftsführung, in schriftlicher oder elektronischer Form zu senden.

    Der Vorstand ist beschlussfähig, wenn mindestens 50\% seiner Mitglieder anwesend oder durch schriftliche Bevollmächtigung eines anwesenden anderen Vorstandsmitglieds vertreten sind. Bei der Beschlussfassung entscheidet die Mehrheit der abgegebenen gültigen Stimmen.

    Der Vorstand kann für seine Tätigkeit eine angemessene Vergütung erhalten. Die maximale Höhe der Vergütung der Vorstandsmitglieder ist vorher durch die Mitgliederversammlung festzulegen. Näheres zu Aufwandsentschädigungen innerhalb des Vereins regelt die Vereinsordnung.

    \Clause{title={Mitgliederversammlung}}
    Eine ordentliche Mitgliederversammlung ist mindestens einmal jährlich einzuberufen.

    Eine außerordentliche Mitgliederversammlung ist einzuberufen, wenn es das Vereinsinteresse erfordert oder wenn die Einberufung von 10\% der Vereinsmitglieder schriftlich und unter Angabe des Zweckes und der Gründe verlangt wird.

    Die Einberufung der Mitgliederversammlung erfolgt schriftlich oder elektronisch an die dem Verein bekannte Adresse des Mitglieds durch zwei Mitglieder des Vorstandes unter Wahrung einer Einladungsfrist von mindestens 2 Wochen bei gleichzeitiger Bekanntgabe der Tagesordnung. Bei elektronischer Einladung gilt das Datum des Versandes der E-Mail laut Serverdaten der Korrespondenz-Adresse des Vorstandes. Das Einladungsschreiben gilt dem Mitglied als zugegangen, wenn es an die letzte vom Mitglied des Vereins schriftlich bekannt gegebene Adresse gerichtet ist. Näheres regelt die Geschäftsordnung des Vorstands.

    Die Mitgliederversammlung als das oberste beschlussfassende Vereinsorgan ist grundsätzlich für alle Aufgaben zuständig, sofern bestimmte Aufgaben gemäß dieser Satzung nicht einem anderen Vereinsorgan übertragen wurden. Ihr sind insbesondere die Jahresrechnung und der Jahresbericht zur Beschlussfassung über die Genehmigung und die Entlastung des Vorstandes schriftlich vorzulegen. Die Mitgliederversammlung entscheidet z. B. auch über
    \begin{enumerate}[(a)]
        \item die Grundsätze von Gebührenbefreiungen,
        \item Aufgaben des Vereins,
        \item An- und Verkauf sowie Belastung von Grundbesitz,
        \item Beteiligung an Gesellschaften,
        \item Aufnahme von Darlehen ab 1.000 €,
        \item Genehmigung aller Geschäftsordnungen für den Vereinsbereich,
        \item Mitgliedsbeiträge,
        \item Satzungsänderungen,
        \item Auflösung des Vereins.
    \end{enumerate}

    Jede satzungsmäßig einberufene Mitgliederversammlung wird als beschlussfähig anerkannt ab der Teilnahme von 15\% aller stimmberechtigten Mitglieder. Ist die Mitgliederversammlung nicht beschlussfähig, so ist vor Ablauf von zwei Wochen seit dem Versammlungstag eine weitere Mitgliederversammlung mit derselben Tagesordnung einzuberufen. Die weitere Versammlung hat spätestens vier Wochen nach dem ersten Versammlungstag stattzufinden. Die neue Versammlung ist ohne Rücksicht auf die Zahl der erschienenen Mitglieder beschlussfähig. Die Einladung zu jener Versammlung muss einen Hinweis auf die erleichterte Beschlussfähigkeit enthalten.

    Die Mitgliederversammlung fasst ihre Beschlüsse mit einfacher Mehrheit, soweit dies nicht anders in dieser Satzung der Vereinsordnung oder per Gesetz geregelt ist. Bei Stimmengleichheit entscheidet die Stimme des/der Vorstandsvorsitzende/n. Beschlüsse sind im Protokoll festzuhalten.
    \label{abstimmungsart}

    Alle weiteren der Einberufung und den Ablauf einer Mitgliederversammlung betreffenden Punkte regelt die Vereinsordnung.

    \Clause{title={Satzungsänderung}}
    Für Satzungsänderungen ist eine Zwei-Drittel-Mehrheit der erschienenen Vereinsmitglieder erforderlich. Über Satzungsänderungen kann in der Mitgliederversammlung nur abgestimmt werden, wenn auf diesen Tagesordnungspunkt bereits in der Einladung zur Mitgliederversammlung hingewiesen wurde und der Einladung sowohl der bisherige als auch der vorgesehene neue Satzungstext beigefügt worden waren.

    Satzungsänderungen, die von Aufsichts-, Gerichts- oder Finanzbehörden aus formalen Gründen verlangt werden, kann der Vorstand von sich aus vornehmen. Diese Satzungsänderungen müssen allen Vereinsmitgliedern alsbald schriftlich mitgeteilt werden.

    \Clause{title={Beurkundung von Beschlüssen}}
    Die in Vorstandssitzungen und in Mitgliederversammlungen erfassten Beschlüsse sind schriftlich niederzulegen und vom Vorstand zu unterzeichnen.

    \Clause{title={Auflösung des Vereins und Vermögensbindung}}
    Für den Beschluss, den Verein aufzulösen, ist eine 3/4-Mehrheit der in der Mitgliederversammlung anwesenden Mitglieder erforderlich. Der Beschluss kann nur nach rechtzeitiger Ankündigung in der Einladung zur Mitgliederversammlung gefasst werden.

    Bei Auflösung oder Aufhebung des Vereins oder bei Wegfall steuerbegünstigter Zwecke fällt das Vermögen des Vereins an eine juristische Person des öffentlichen Rechts oder eine andere steuerbegünstigte Körperschaft zwecks Verwendung für die Förderung von Kunst und Kultur, der Förderung des Naturschutzes und der Landschaftspflege im Sinne des Bundesnaturschutzgesetzes und der Naturschutzgesetze der Länder, des Umweltschutzes, einschließlich des Klimaschutzes, des Küstenschutzes und des Hochwasserschutzes, sowie der Förderung der Gleichberechtigung von Frauen und Männern nach \S 52 Abs. 2 Nr. 5, 8 und 18.

\end{contract}

\section{Weiteres}
\begin{contract}

    \Clause{title={Datenschutz}}
    Im Rahmen der Mitgliederverwaltung werden von den Mitgliedern folgenden Daten erhoben:
    \begin{enumerate}[(a)]
        \item Name;
        \item Vorname;
        \item Geburtsdatum;
        \item Anschrift;
        \item private E-Mail Adresse;
    \end{enumerate}

    Die Daten werden im Rahmen der Mitgliedschaft verarbeitet und gespeichert. Der Verein veröffentlicht Daten seiner Mitglieder auf der Homepage oder auf Plakaten nur, wenn das Mitglied schriftlich zugestimmt hat.

    \Clause{title={Salvatorische Klausel}}

    Soweit einzelne Bestimmungen dieser Satzung ganz oder teilweise nicht rechtswirksam sind, wird hierdurch die Gültigkeit der übrigen Bestimmungen nicht berührt. Anstelle der unwirksamen Bestimmung soll im Wege der Anpassung eine andere angemessene Regelung gelten, die wirtschaftlich dem am nächsten kommt, was die Beschließenden gewollt haben oder vereinbart hätten, wenn sie die Unwirksamkeit der Regelung bedacht hätten.


\end{contract}

\newpage
\section{Unterschriften}
\foreach \n in {1,...,14}{
        \parbox{\textwidth}{
            \vspace{5ex}
            \parbox{5cm}{
                \rule{5cm}{0.3pt}\\
                Vor- und Nachname
            }
            \hfill
            \parbox{3cm}{
                \rule{3cm}{0.3pt}\\
                Datum
            }
            \hfill
            \parbox{6cm}{
                \rule{6cm}{0.3pt}\\
                Unterschrift
            }
        }
    }

\end{document}
