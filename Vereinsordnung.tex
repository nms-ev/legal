\documentclass[fontsize=12pt,parskip=half] {scrartcl}
\usepackage{lmodern}
\usepackage{textcomp}
\usepackage[shortlabels]{enumitem}
\usepackage[clausemark=forceboth,juratotoc, juratocnumberwidth=2.5em]{scrjura}
\usepackage{eurosym}
\usepackage{csquotes}
\usepackage{pgffor}

\MakeOuterQuote{"}

\newcommand\name{nms e.V.}
 
\DeclareNewJuraEnvironment{beitrag}[]{}{}
\DeclareNewJuraEnvironment{wahl}[]{}{}
\DeclareNewJuraEnvironment{gfvorstand}[]{}{}
\DeclareNewJuraEnvironment{gfmitglieder}[]{}{}
\DeclareNewJuraEnvironment{finanz}[]{}{}
 
\begin{document}

\subject{Vereinsordnung}
\title{\name}
\subtitle{Vereinsordnung für den \name}
\date{\today\\ \version}
\maketitle

\tableofcontents

\appendix

\section{Beitragsordnung}
\begin{beitrag}

  \Clause{title={Grundsatz}}
  Diese Beitragsordnung ist nicht Bestandteil der Satzung. Sie regelt die Beitragsverpflichtungen der Mitglieder sowie die Gebühren und Umlagen. Sie kann nur von der Mitgliederversammlung des Vereins geändert werden.

  \Clause{title={Beschlüsse}}
  Die Mitgliederversammlung beschließt die Höhe des Beitrags, die Aufnahmegebühr und Umlagen. Der Vorstand legt die Gebühren fest.

  Die festgesetzten Beträge werden zum 1. Januar des folgenden Jahres erhoben, in dem der Beschluss gefasst wurde. Durch Beschluss der Mitgliederversammlung kann auch ein anderer Termin festgelegt werden.

  \Clause{title={Beiträge}}

  \begin{center}
    \begin{tabular}{ |c|l|r| }
      \hline
      Klasse & Mitgliedsform               & Beitragshöhe     \\
      \hline \hline
      1      & Stimmberechtigte Mitglieder & 84,00\euro \\
      2      & Fördermitglieder            & $\ge$ 20,00\euro \\
      \hline
    \end{tabular}
  \end{center}

  Der Mitgliedsbeitrag muss bis spätestens 15.01. eines jeden Jahres auf das Beitragskonto des Vereins entrichtet werden.

  Bei Mahnungen werden Mahngebühren von 5\euro{} pro Mahnung erhoben.

  Erfolgt der Vereinseintritt nach dem 30.06. erfolgt eine Berechnung von 50\% des Beitragssatzes.

  \Clause{title={Vereinskonto}}

  \begin{center}
    \begin{tabular}{ |ll| }
      \hline
      Inhaber: & nms e.V.                    \\
      \hline
      IBAN:    & DE56 8601 0090 0993 8629 03 \\
      \hline
      BIC:     & PBNKDEFF                    \\
      \hline
    \end{tabular}
  \end{center}

  Überweisung auf andere Konten sind nicht zulässig und werden nicht als Zahlungen anerkannt.

  \Clause{title={Vereinsaustritt}}
  Ein Vereinsaustritt nach \S 8 der Vereinssatzung ist bis zum 30.11. des Jahres zum Jahresende möglich.

\end{beitrag}

\section{Wahlordnung}
\begin{wahl}
  \label{wahlordnung}

  \Clause{title={Wahlvorschläge und Wahlleitung }}
  Wahlvorschläge können gemacht werden
  \begin{enumerate}[(a)]
    \item durch den Vorstand
    \item durch die Mitglieder
  \end{enumerate}

  Wahlvorschläge des Vorstands und etwa schon vorliegende Vorschläge von Mitgliedern werden mit der Einladung zur Mitgliederversammlung mitgeteilt.

  Die Wahl wird von dem:der Vorstandsvorsitzenden oder im Falle dessen:deren Verhinderung von einem weiteren Mitglied des Vorstands geleitet.

  Spätestens zu Beginn der Wahl gibt der:die Wahlleiter:in die Wahlvorschläge bekannt.

  \Clause{title={Wahlverfahren}}
  Gewählt wird geheim und schriftlich auf vorbereiteten Stimmzetteln, auf denen die Kandidat:innen in alphabetischer Reihenfolge aufgeführt werden. Jedes Mitglied hat eine Stimme. Der:die Kandidat:in sind gewählt, die die Mehrheit der abgegebenen gültigen Stimmzettel erreicht haben.

  Werden auf Stimmzetteln mehr als ein Name angekreuzt, oder enthält der Stimmzettel sonstige Zusätze, so ist er ungültig.

  Auf Antrag kann ein offenes Wahlverfahren per Handzeichen durchgeführt werden. Dieses ist einstimmig von der Mitgliederversammlung zu beschließen.

  \Clause{title={Annahme der Wahl}}
  Der:die Wahlleiter:in gibt das Ergebnis der Wahl bekannt. Ist der:die Gewählte bei Bekanntgabe des Wahlergebnisses nicht anwesend, wird er von dem:der Vorstandsvorsitzenden schrifltich von seiner Wahl benachrichtigt.
  \label{wahl:annahme}

  Die anwesenden Gewählten haben sich sofort, Abwesende unverzüglich nach Zugang der Mitteilung gemäß \ref{wahl:annahme}, über die Annahme zu erklären.

\end{wahl}

\section{Geschäftsordnung des Vorstandes}

\begin{gfvorstand}

  \Clause{title={Sitzungen}}
  Vorstandssitzungen finden regelmäßig mindestens ein mal im Quartal statt.

  \Clause{title={Vertraulichkeit / Öffentlichkeit}}
  Die Sitzungen des Vorstandes sind nicht öffentlich.

  Der Vorstand kann mit einfacher Mehrheit über die Zulassung weiterer Personen zur Sitzung entscheiden.

  Die im Rahmen der Vorstandssitzung behandelten Inhalte sind vertraulich zu behandeln.

  Die genemigte Niederschrift ist für alle Mitglieder einsehbar.

  \Clause{title={Sitzungsleitung}}
  Die Sitzungen des Vorstands werden von dem:der Vorsitzenden geleitet. Sollte der:die Vorsitzende verhindert sein, so wählt der Vorstand mit einfacher Mehrheit eine Vertretung für die Sitzungsleitung.

  \Clause{title={Beschlussfähigkeit}}

  Der Vorstand ist beschlussfähig, wenn mehr als die Hälfte der Vorstandsmitglieder anwesend sind.

  Die Beschlussfähigkeit ist zu Beginn der Sitzung von dem:der Sitzungsleitenden festzustellen.

  \Clause{title={Beratungsgegenstand}}

  Gegenstand der Beratung sind nur die in der Tagesordnung festgelegten Beratungspunkte.

  In dringenden Fällen können weitere Tagesordnungspunkte zugelassen werden. Voraussetzung dafür ist die einfache Mehrheit der im Sitzungstermin anwesenden Vorstandsmitglieder.

  \Clause{title={Abstimmung}}

  Zur Abstimmung sind nur die in den Vorstandssitzungen anwesenden Mitglieder des Vorstandes berechtigt. Eine Stimmrechtsübertragung ist ausgeschlossen.

  Abstimmungen erfolgen in der durch die:den Sitzungsleitende:n bestimmten Form (Handzeichen, Zuruf, schriftliche Abstimmung).

  Der Vorstand entscheidet über Anträge mit einfacher Mehrheit. Im Falle der Stimmengleichheit wird die Abstimmung nach nochmaliger Beratung wiederholt. Sollte im Wiederholungsfall eine erneute Stimmengleichheit festgestellt werden, so gilt der Antrag als abgelehnt.

  \Clause{title={Niederschrift}}

  Der Ablauf einer jeden Vorstandssitzung ist durch den:die Protokollant:in schriftlich festzuhalten.

  Das gefertigte Sitzungsprotokoll ist von dem:der Sitzungsleiter:in und dem:der Protokollant:in zu unterzeichnen.

  Jedem Vorstandsmitglied ist eine Abschrift des Sitzungsprotokolls zur Verfügung zu stellen.

  Gegen den Inhalt des Protokolls kann jedes Vorstandsmitglied innerhalb einer zweiwöchigen Frist schriftlich oder fernmündlich Einwendungen erheben. Über Einwendungen wird in der nächsten Vorstandssitzung entschieden. Sollte bis zum Ablauf der Frist keine Einwendungen erhoben werden, so gilt das Sitzungsprotokoll als genehmigt.

\end{gfvorstand}

\section{Geschäftsordnung der Mitgliederversammlung}

\begin{gfmitglieder}

  \Clause{title={Einberufung}}

  \newcommand{\mitgliederversammlung}{\S 15}

  Der Anlass zur Einberufung einer Mitgliederversammlung richtet sich nach \mitgliederversammlung{} der Satzung.

  Eine ordentliche Mitgliederversammlung ist den Mitgliedern zwei Wochen vorher durch schriftliche Benachrichtigung anzukündigen.

  Die vorläufige Tagesordnung stellt der:die Vorstandsvorsitzende:r auf. Schriftlichen Anträgen der Mitglieder auf Aufnahme von Beratungsgegenständen in die Tagesordnung ist stattzugeben, wenn die Anträge eine Woche vor der Versammlung dem Vorstand vorliegen.
  \label{einberufung:tagesordnung}

  Eine außerordentliche Mitgliederversammlung ist gemäß \mitgliederversammlung{} ordnungsgemäß einzuberufen. Die schriftliche Ladung der Mitglieder erfolgt eine Woche vor der Versammlung unter Angabe der Tagesordnung. Anträge der Mitglieder auf Aufnahme von Beratungsgegenständen in die Tagesordnung \ref{einberufung:tagesordnung} sollen in diesen Fällen möglichst mit der Einberufung, spätestens aber drei Tage vor der Versammlung versandt werden.

  \Clause{title={Öffentlichkeit und Teilnahme}}

  Die Mitgliederversammlung ist öffentlich. Die Öffentlichkeit ist auszuschließen, wenn es die Mehrheit der erschienenen Mitglieder mit einfacher Mehrheit der abgegebenen gültigen Stimmen beschließt.

  Gäste können an der Mitgliederversammlung teilnehmen. Sie haben kein Rede- und Stimmrecht.

  \Clause{title={Leitung der Mitgliederversammlung}}

  Der:die Vorstandsvorsitzende eröffnet, leitet und schließt die Mitgliederversammlung. Sie:er wird bei Verhinderung von einem anderen Vorstandsmitglied vertreten.

  Sind alle Mitglieder des Vorstands verhindert, so wählt die Mitgliederversammlung mit der einfachen Mehrheit der erschienenen Mitglieder eine:n Leiter:in.

  Bei Gegenständen, Beratungen und Abstimmungen, die den:die Versammlungsleiter:in selbst in Person betreffen, muss diese:r die Versammlungsleitung abgeben. In diesem Fall hat die Mitgliederversammlung für diesen Tagesordnungspunkt eine:n Vertreter:in zu wählen.

  \Clause{title={Eröffnung der Mitgliederversammlung}}

  Nach der Eröffnung der Mitgliederversammlung stellt der:die Sitzungsleiter:in die ordnungsgemäße Einberufung fest. Anhand der Anwesenheitsliste wird die Zahl der anwesenden stimmberechtigten Mitglieder festgestellt und sodann auch die Beschlussfähigkeit der Versammlung.

  Zu Beginn der Mitgliederversammlung wird von dem:der Sitzungsleiter:in ein:e Protokollführer:in bestimmt.

  \Clause{title={Tagesordnung}}

  Nach der Eröffnung wird die Tagesordnung bekannt gegeben.

  Die Mitgliederversammlung kann mit der einfachen Mehrheit der erschienenen Mitglieder die Tagesordnung ändern.

  \Clause{title={Wortmeldungen und Redeordnung}}

  Der:die Versammlungsleiter:in erteilt den Mitgliedern in der Reihenfolge ihrer Wortmeldungen das Wort, wenn für den Beratungsgegenstand, der eröffnet ist, die Aussprache erfolgt.

  Die Redezeit kann von dem:der Leiter:in begrenzt werden.

  \Clause{title={Ordnungsmaßnahmen des Leitenden}}

  Unqualifizierte Äußerungen hat der:die Leiter:in zu unterbinden. Bei Wiederholung ist dem:der Störer:in das Wort zu entziehen.

  Der:die Leiter:in hat auch die Möglichkeit, Störer:innen aus dem Saal zu verweisen oder andere geeignete Maßnahmen zu treffen.

  Beteiligen sich mehrere Teilnehmer:innen an der Störung der Versammlung, so kann der:die Leiter:in die Versammlung auf Zeit unterbrechen.

  Beim Ausschluss von Gäst:innen wegen grober Ordnungsstörung macht der:die Leiter:in von dem ihm:ihr übertragenen Hausrecht Gebrauch.

  \Clause{title={Abstimmungen}}

  Über jeden Beratungsgegenstand muss gesondert abgestimmt werden, es sei denn, dass Gegenstände verbunden worden sind.

  Während des Abstimmungsverfahrens sind nur noch solche Anträge zulässig, die redaktionellen Inhalt haben.

  Jeder Antrag ist vor der Abstimmung nochmals bekannt zu geben. Abstimmungsfragen sind so zu stellen, dass sie mit \enquote{Ja} oder \enquote{Nein} beantwortet werden können.

  Liegen zu einem Beschlussgegenstand mehrere Anträge vor, so ist über den weitestgehenden zuerst abzustimmen. Bestehen Zweifel, welcher Antrag der weitestgehende ist, so wird hierüber durch vorherige Abstimmung ohne Aussprache entschieden.

  \Clause{title={Abstimmungsverfahren}}

  Abstimmungen erfolgen entweder durch Handzeichen (offene Abstimmung) oder schriftlich durch Stimmzettel (geheime Abstimmung).

  Grundsätzlich wird offen abgestimmt. Geheim ist abzustimmen, wenn mehrere Wahlvorschläge vorliegen oder wenn die einfache Mehrheit der erschienenen stimmberechtigten Mitglieder dies verlangt.

  \Clause{title={Abstimmungsmehrheiten und -ergebnis}}

  Bei Abstimmungen und Wahlen genügt grundsätzlich die einfache Mehrheit der erschienenen Mitglieder. Die erforderliche Mehrheit errechnet sich ausschließlich aus den abgegebenen gültigen \enquote{Ja} und \enquote{Nein} Stimmen. Stimmenthaltungen werden ebenso wenig wie ungültige Stimmen berücksichtigt.

  Die Änderung des Vereinszwecks kann nur mit zwei Drittel Mehrheit aller Mitglieder beschlossen werden. Die schriftliche Zustimmung der in der Mitgliederversammlung nicht erschienenen Mitglieder kann innerhalb eines Monats gegenüber dem Vorstand erklärt werden.

  Der:die Leiter:in gibt das Abstimmungsergebnis der Versammlung bekannt. Das Ergebnis ist genau von dem:der Protokollführer:in in die Niederschrift über die Versammlung aufzunehmen.

  \Clause{title={Wahlen}}

  Wahlen können nur durchgeführt werden, wenn sie als Beschlussgegenstand auf der Tagesordnung enthalten sind.

  Näheres zu dem Ablauf einer Wahl regelt die Wahlordnung in Abscnitt \ref{wahlordnung}.

  \Clause{title={Versammlungsprotokoll}}

  Über jede Mitgliederversammlung ist ein Protokoll zu führen, das die wesentlichen Ergebnisse enthalten muss.

  Das Protokoll ist von dem:der Leiter:in und dem:der Protokollführer:in zu unterzeichnen.

  Auf Verlangen müssen während oder nach der Versammlung abgegebene Erklärungen in das Protokoll aufgenommen werden.

  Einwendungen gegen das Protokoll sind bei dem:der Leiter:in innerhalb eines Monats nach Bekanntgabe des Protokolls schriftlich zu erheben.

\end{gfmitglieder}

\section{Finanzordnung}

\begin{finanz}

  \Clause{title={Grundsätze Wirtschaftlichkeit und Sparsamkeit}}

  Der Verein ist nach den Grundsätzen der Wirtschaftlichkeit zu führen, das heißt, die Aufwendungen müssen in einem wirtschaftlichen Verhältnis zu den erzielten und erwarteten Erträgen stehen.

  Für den Verein gilt generell das Kostendeckungsprinzip.

  Die Mittel des Vereins dürfen nur für die satzungsmäßigen Zwecke verwendet werden. Die Mitglieder erhalten in ihrer Eigenschaft als Mitglieder hieraus keine Zuwendungen.

  Es darf keine Person durch Ausgaben, die dem Zweck des Vereins fremd sind oder durch unverhältnismäßig hohe Vergütungen begünstigt werden.

  \Clause{title={Jahresabschluss}}

  Im Jahresabschluss müssen alle Einnahmen und Ausgaben des Gesamtvereins für das abgelaufene Geschäftsjahr nachgewiesen werden. Im Jahresabschluss muss darüber hinaus eine Schulden- und Vermögensübersicht enthalten sein.

  Der Jahresabschluss wird nach Fertigstellung öffentlich zugänglich gemacht.

  \Clause{title={Verwaltung der Finanzmittel}}

  Alle Finanzgeschäfte werden über die Vereinshauptkasse abgewickelt.

  Der Vorstand verwaltet die Vereinshauptkasse.

  Zahlungen aus der Vereinshauptkasse werden nur geleistet, wenn sie nach \ref{zahlungsverkehr} dieser Finanzordnung ordnungsgemäß ausgewiesen sind und ausreichende Finanzmittel zur Verfügung stehen.

  \Clause{title={Erhebung und Verwendung der Finanzmittel}}

  Alle Mitgliedsbeiträge werden vom Gesamtverein erhoben und verbucht.

  Überschüsse aus Veranstaltungen werden über die Vereinshauptkasse verbucht.

  Die Finanzmittel sind entsprechend der Finanzordnung zu verwenden.

  % TODO: Ask verein about diz
  ..\% der Überschüsse aus Veranstaltungen sind nach Jahresabschluss stets für den Spendenzweck zu verwenden. (Frequenz des Spendens, Höhe, Veranstaltungsbasiert oder nicht)

  \Clause{title={Zahlungsverkehr}}
  \label{zahlungsverkehr}

  Der gesamte Zahlungsverkehr wird über die Vereinshauptkasse und wenn möglich bargeldlos abgewickelt.

  Über jede Einnahme und Ausgabe muss ein Beleg vorhanden sein. Der Beleg muss den Tag der Ausgabe, den zu zahlenden Betrag, die Mehrwertsteuer und den Verwendungszweck enthalten.

  Bei Gesamtabrechnungen muss auf dem Deckblatt die Zahl der Unterbelege vermerkt werden.

  Die bestätigten Rechnungen sind dem Vorstand, unter Beachtung von Skonto-Fristen rechtzeitig zur Begleichung einzureichen.

  Wegen des Jahresabschlusses sind Barauslagen zum 30.12. des auslaufenden Jahres mit dem Vorstand abzurechnen.

  Zur Vorbereitung von Veranstaltungen ist es dem Vorstand gestattet, Vorschüsse in Höhe des zu erwartenden Bedarfs zu gewähren. Diese Vorschüsse sind spätestens 2 Monate nach Beendigung der Veranstaltung abzurechnen.

  \Clause{title={Spenden}}

  Spenden müssen mit der Angabe der Zweckbestimmung dem \name zur Weiterleitung an den Verein überwiesen werden.

  Spenden kommen dem Gesamtverein zugute.

  \Clause{title={Inventar}}

  Zur Erfassung des Inventars ist von der Geschäftsstelle eine Inventar-Liste anzulegen.

  Es sind alle Gegenstände mit einem Wert von mehr als 100\euro{} aufzunehmen, die nicht zum Verbrauch bestimmt sind.Die Inventar-Liste muss enthalten:
  \begin{enumerate}[(a)]
    \item Anschaffungsdatum
    \item Bezeichnung des Gegenstandes
    \item Anschaffungswert
    \item Aufbewahrungsort
    \item (Gegenstände, die ausgesondert werden, sind mit einer kurzen Begründung anzuzeigen.)
  \end{enumerate}

  Unbrauchbares bzw. überzähliges Gerät und Inventar ist möglichst gewinnbringend zu veräußern. Der Erlös muss der Vereinshauptkasse zugeführt werden.

  Über verschenkte Gegenstände ist ein Beleg vorzulegen.

  \Clause{title={Zuschüsse}}

  Nicht zweckgebundene Zuschüsse werden im Rahmen der Mitgliederversammlung verteilt.

\end{finanz}

\end{document}

